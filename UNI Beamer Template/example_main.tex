\documentclass{beamer}
\usepackage{amsfonts,amsmath,oldgerm}
\usetheme{sintef}
\usepackage{xeCJK}
\usepackage{listings}
\usepackage{xcolor}
\usepackage{amsmath}
\definecolor{codegray}{rgb}{0.95,0.95,0.95}
\usepackage{algorithm}
\usepackage{algorithmic}
\usepackage[linesnumbered,ruled,vlined]{algorithm2e}

\usepackage{minted}

\newcommand{\testcolor}[1]{\colorbox{#1}{\textcolor{#1}{test}}~\texttt{#1}}

\usefonttheme[onlymath]{serif}

\titlebackground*{assets/background}

\newcommand{\hrefcol}[2]{\textcolor{cyan}{\href{#1}{#2}}}

\title{Título}
\subtitle{Subtítulo}
\course{Análisis y diseño de Algoritmos}
\author{André Pacheco}
% \IDnumber{1234567}
\date{ --, --, 2024}


% Configuración adicional para algorithm2e
\SetAlFnt{\scriptsize}
\SetAlCapFnt{\footnotesize}
\SetAlCapNameFnt{\footnotesize}
\SetAlCapHSkip{0pt}
\IncMargin{1em}

% Configuración en español
\SetKwInput{KwIn}{Entrada}
\SetKwInput{KwOut}{Salida}
\SetKwFor{While}{mientras}{hacer}{fin mientras}
\SetKwIF{If}{ElseIf}{Else}{si}{entonces}{sino si}{sino}{fin si}
\SetKwFor{For}{para}{hacer}{fin para}
\SetKwRepeat{Repeat}{repetir}{hasta que}
\SetKwBlock{Begin}{inicio}{fin}

\begin{document}
\maketitle

\section{Método de Bisección para Optimización}

\begin{frame}
\frametitle{Descripción}
\begin{itemize}
    \item Busca el mínimo de una función unimodal en un intervalo cerrado
    \item Divide repetidamente el intervalo, seleccionando el subintervalo con el mínimo
    \item Simple y robusto, aunque relativamente lento
\end{itemize}
\textit{Observación}: Análogo a la búsqueda binaria, pero buscando el punto mínimo de la función
\end{frame}

\begin{frame}
\frametitle{Intuición del Algoritmo}
\begin{itemize}
    \item Se basa en la naturaleza de las funciones unimodales
    \item Divide el intervalo en tres partes usando dos puntos internos
    \item Evalúa la función en estos puntos y en el punto medio
    \item Compara los valores para determinar en qué tercio \textit{no} está el mínimo
    \item Descarta ese tercio y repite el proceso con el intervalo reducido
\end{itemize}
\end{frame}


\begin{frame}
\frametitle{Algoritmo}
\begin{algorithm}[H]
\SetAlgoLined
\KwIn{$f$, $a$, $b$, $\epsilon$}
\KwOut{Punto mínimo aproximado}
\While{$b - a > \epsilon$}{
    $m \gets (a + b) / 2$\;
    $x_1 \gets (a + m) / 2$\;
    $x_2 \gets (m + b) / 2$\;
    \eIf{$f(x_1) < f(m)$}{
        $b \gets m$\;
    }{
        \eIf{$f(x_2) < f(m)$}{
            $a \gets m$\;
        }{
            $a \gets x_1$\;
            $b \gets x_2$\;
        } % OJO: la diapositiva SOLO admite hasta 10 líneas de código por defecto antes de que se pase del espacio admitido
    }
}
\Return{$(a + b) / 2$}\;
\caption{Método de Bisección para Optimización}
\end{algorithm}
\end{frame}

\begin{frame}[fragile]
\frametitle{Ejemplo Paso a Paso (1/2)}
Función: $f(x) = (x - 2)^2 + 1$
\begin{enumerate}
    \item \textbf{Inicialización}:
    \begin{itemize}
        \item $a = 0$, $b = 3$
    \end{itemize}
    \item \textbf{Primera Iteración}:
    \begin{itemize}
        \item $m = 1.5$, $x_1 = 0.75$, $x_2 = 2.25$
        \item $f(x_1) = 2.5625$, $f(m) = 1.25$, $f(x_2) = 1.0625$
        \item Actualizar: $a = 1.5$, $b = 3$
    \end{itemize}
\end{enumerate}
\end{frame}

\begin{frame}
\frametitle{Ejemplo Paso a Paso (2/2)}
\begin{enumerate}\setcounter{enumi}{2}
    \item \textbf{Segunda Iteración}:
    \begin{itemize}
        \item $m = 2.25$, $x_1 = 1.875$, $x_2 = 2.625$
        \item $f(x_1) = 1.015625$, $f(m) = 1.0625$, $f(x_2) = 1.390625$
        \item Actualizar: $a = 1.5$, $b = 2.25$
    \end{itemize}
    \item \textbf{Tercera Iteración}:
    \begin{itemize}
        \item $m = 1.875$, $x_1 = 1.6875$, $x_2 = 2.0625$
        \item $f(x_1) = 1.099609375$, $f(m) = 1.015625$, $f(x_2) = 1.00390625$
        \item Actualizar: $a = 1.875$, $b = 2.25$
    \end{itemize}
\end{enumerate}


El proceso continúa hasta alcanzar la precisión deseada.
\end{frame}


\begin{frame}[fragile]
\frametitle{Código en Python (1/2)}
\begin{minted}[fontsize=\footnotesize, bgcolor=codegray]{python}
def biseccion_optimizacion(f, a, b, epsilon):
    while b - a > epsilon:
        m = (a + b) / 2
        x1 = (a + m) / 2
        x2 = (m + b) / 2
        if f(x1) < f(m):
            b = m
        elif f(x2) < f(m):
            a = m
        else:
            a, b = x1, x2
    return (a + b) / 2
\end{minted}
\end{frame}

\begin{frame}[fragile]
\frametitle{Código en Python (2/2)}
\begin{minted}[fontsize=\footnotesize, bgcolor=codegray]{python}
# Ejemplo de uso
def f(x):
    return (x - 2)**2 + 1

a, b = 0, 3
epsilon = 0.001
minimo = biseccion_optimizacion(f, a, b, epsilon)
print(f"Punto mínimo aprox.: {minimo}")
print(f"Valor mínimo aprox.: {f(minimo)}")
\end{minted}

\begin{minted}[fontsize=\footnotesize, bgcolor=codegray]{python}
El punto mínimo aproximado es: 1.9998779296875
El valor mínimo aproximado es: 1.0000000149011612
\end{minted}
\end{frame}

\backmatter
\end{document}
